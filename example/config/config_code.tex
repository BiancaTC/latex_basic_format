%%%%%%%% code inclusion:
\usepackage{listings}
\usepackage{setspace}

\definecolor{Color_Background}{rgb}{1,1,1}
\definecolor{Color_Code}{rgb}{1,0,0}
\definecolor{Color_Decorators}{rgb}{0.5,0.5,0.5}
\definecolor{Color_Numbers}{rgb}{0.36,0.14,0.43}
\definecolor{Color_Comments}{rgb}{0.11,0.6,0.}
\definecolor{Color_Strings}{rgb}{0.81,0.68,0.}

\definecolor{Color_Keywords_1}{rgb}{0.4,0.4,0.4}
\definecolor{Color_Keywords_2}{rgb}{0.05,0.,0.58}
\definecolor{Color_Keywords_3}{rgb}{1,0.51,0}
\definecolor{Color_Keywords_4}{rgb}{0.78,0.46,0.9}
\definecolor{Color_Keywords_5}{rgb}{0.49,0,0.81}
\definecolor{Color_Operators}{rgb}{1,0,0}


\definecolor{Color_Backquotes}{rgb}{1,0,0}
\definecolor{Color_Classname}{rgb}{1,0,0}
\definecolor{Color_FunctionName}{rgb}{1,0,0}
\definecolor{Color_Matching_Brackets}{rgb}{0.25,0.5,0.5}

%%%%%%%%%%%%%%%%%%%%%%%%% PYTHON




\lstset{
language=Python,
                                            %
%numbering
numbers=left,       % where to put the line-numbers; possible values are (none, left, right)
numberstyle=\footnotesize,     % the style that is used for the line-numbers
numbersep=1em,     % how far the line-numbers are from the code
stepnumber = 1, % the step between two line-numbers. If it's 1, each line will be numbered
                                            %
%font style
xleftmargin=1em,
framextopmargin=2em,
framexbottommargin=2em,
showspaces=false,                % show spaces everywhere adding particular underscores; it overrides 'showstringspaces'
showtabs=false,             % show tabs within strings adding particular underscores
showstringspaces=false,       % underline spaces within strings only
frame=l,       % adds a frame around the code
tabsize=4,      % sets default tabsize to x spaces
breaklines=true,                            % allows for breaking lines of the code
breakatwhitespace = false,                  % sets if automatic breaks should only happen at whitespace
                                            %
% Basic
basicstyle=\ttfamily\tiny\setstretch{1},    % the size of the fonts that are used for the code
backgroundcolor=\color{Color_Background},         % choose the background color; you must add \usepackage{color} or \usepackage{xcolor}
                                            %
% Comments
commentstyle=\color{Color_Comments},      % comment style
                                            %
% Strings do not work
%stringstyle=\color{Color_Strings},                  %string style
                                            %
% DocStrings do not work
% I was not able to successfully untangled comments from docstrings.
%moredelim=**[s][\color{gray}]{"""}{"""},    %adding the """
%moredelim=**[s][\color{gray}]{'''}{'''},    %adding the '''
                                            %
% Operators
otherkeywords={!,!=,~,$,$,*,\&,+,-,^,\%,\%/\%,\%*\%,\%\%,<-,<<-,/},
keywordstyle=\color{Color_Keywords_2},
                                            %
                                            %
                                            %
% keywords_1 : BOOL
                                            %
emph={fsdjskq}, % add keywords to a list
emphstyle={\color{Color_Keywords_1}},      % keyword style
                                            %
                                            %
% keywords_1 : BOOL
                                            %
emph={[10]False,True,None}, % add keywords to a list
emphstyle={[10]\color{Color_Keywords_1}},      % keyword style
                                            %
                                            %
% keywords_2 : CLASSICS
                                            %
emph={[2]import,include,from,def,for,
while,if,is,in,elif,else,not,and,or,print,break,
continue,return,as,del,except,exec,global,from,
finally,global,import,lambda,inline,pass,print,
Exception,raise,try,assert}, % add keywords to a list
emphstyle={[2]\color{Color_Keywords_2}\bfseries},      % keyword style
                                            %
                                            %
% keywords_3 : TYPES AND UTILITIES
                                            %
emph={[3]self,class,object,type,isinstance,copy,deepcopy,
zip,enumerate,list,set,len,dict,tuple,range,
append,execfile,real,imag,reduce,str,repr}, % add keywords to a list
emphstyle={[3]\color{Color_Keywords_3}\bfseries},      % keyword style
                                            %
                                            %
% keywords_4 : ALG LIB
                                            %
emph={[4]ode,fsolve,sqrt,exp,sin,cos,
pi,array,norm,solve,dot,arange,linspace,isscalar,max,
sum,flatten,shape,reshape,find,any,all,abs,plot,legend,quad,
polyval,polyfit,hstack,concatenate,vstack,column_stack,
empty,zeros,ones,rand,vander,grid,pcolor,eig,eigs,
eigvals,svd,qr,tan,det,logspace,roll,min,mean,cumsum,cumprod,
diff,vectorize,lstsq,cla,eye,xlabel,ylabel,squeeze}, % add keywords to a list
emphstyle={[4]\color{Color_Keywords_4}},      % keyword style
                                            %
                                            %
% keywords_5 : CLASS FCT
                                            %
emph={[5]__init__,__add__,__mul__,__div__,__sub__,__call__
,__getitem__,__setitem__,__eq__,__ne__,__nonzero__,__rmul__,
__radd__,__repr__,__str__,__get__,__truediv__,__pow__,__name__
,__future__,__all__}, % add keywords to a list
emphstyle={[5]\color{Color_Keywords_5}\bfseries},      % keyword style
                                            %
                                            %
% simple import decorators etc not so important for code!
emph={@invariant,pylab,numpy,np,scipy,plt,math,bisect,tqdm},
emphstyle={\color{Color_Decorators}\slshape},
}




%%%%%%% changes color for digits

% from https://tex.stackexchange.com/questions/34896/coloring-digits-with-the-listings-package
% second answer:
\newcommand\digitstyle{\color{Color_Numbers}}
\makeatletter
\newcommand{\ProcessDigit}[1]
{%
  \ifnum\lst@mode=\lst@Pmode\relax%
   {\digitstyle #1}%
  \else
    #1%
  \fi
}
\makeatother
\lstset{literate=
    {0}{{{\ProcessDigit{0}}}}1
    {1}{{{\ProcessDigit{1}}}}1
    {2}{{{\ProcessDigit{2}}}}1
    {3}{{{\ProcessDigit{3}}}}1
    {4}{{{\ProcessDigit{4}}}}1
    {5}{{{\ProcessDigit{5}}}}1
    {6}{{{\ProcessDigit{6}}}}1
    {7}{{{\ProcessDigit{7}}}}1
    {8}{{{\ProcessDigit{8}}}}1
    {9}{{{\ProcessDigit{9}}}}1
    {<=}{{\(\leq\)}}1,
    morestring=[b]",
    morestring=[b]',
    morecomment=[l]//,
}
