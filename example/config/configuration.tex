\usepackage[utf8]{inputenc}
\usepackage[T1]{fontenc}

\usepackage[left=2.5cm,right=2.5cm,top=2cm,bottom=3cm]{geometry}%réglages des marges du document selon vos préférences ou celles de votre établissemant
\setlength{\headheight}{15pt}% hauteur de l'entête

\usepackage{array}  %pour les array et binomes de newton 
\usepackage{amsmath,amsfonts,amssymb}%extensions de l'ams pour les mathématiques
\usepackage{amsthm} %pour les théoremes
\usepackage{colortbl}
\usepackage[dvipsnames]{xcolor}
\usepackage{comment} % pour les commentaires
\usepackage{dsfont} %fonction indicatrice
\usepackage{fancybox} %pour shadow box
\usepackage[Rejne]{fncychap}%pour de jolis titres de chapitres voir la doc pour d'autres styles.
\usepackage{fancyhdr}%pour les entêtes et pieds de pages
\usepackage{graphicx}%pour insérer images et pdf entre autres
\graphicspath{{images/}}%pour spécifier le chemin d'accès aux images
\usepackage{lmodern}	%celui ci et le suivant pour les boites
\usepackage{shorttoc}%pour la réalisation d'un sommaire.
\usepackage[most]{tcolorbox}

\usepackage{lipsum} %to test paragraphs
\usepackage{verbatim} %for writing as code
\usepackage{fancyvrb} %this is for reducing size of verbatim
\usepackage{subfig} %for having the figures next to each other
\usepackage{stmaryrd} % pacakge for double brackets, notation for integers.

% packages for tables rotated. and algorithms.
\usepackage[ruled,vlined, linesnumbered]{algorithm2e}
\usepackage{lscape}
\usepackage{rotating}




\usepackage[english]{babel}%pour un document en englais
\usepackage{hyperref}%rend actif les liens, références croisée, toc, ...
		\hypersetup{colorlinks,%
		citecolor=black,%
		filecolor=black,%
		linkcolor=gray,%
		urlcolor=blue} 




		
		
		
%%%%%%%%%%%%%%%%%%%%%%%%%%%%%%%%%%%%%%%%%%%%%%%%%%%%%%%%%%%%%%%%%%%%%%%%%%%%%%%%%%%%%%%%
\makeatletter
\newenvironment{abstract}{%
    \cleardoublepage
    \null\vfil
    \@beginparpenalty\@lowpenalty
    \begin{center}%
      \bfseries \abstractname
      \@endparpenalty\@M
    \end{center}}%
   {\par\vfil\null}
\makeatother


\newenvironment{acknowledgements}{
\renewcommand\abstractname{Acknowledgements}
\begin{abstract}} {\end{abstract}
}


		




%%%%%%%%%%%%%%Remarque
\newtheoremstyle{rq}{}{}{\advance \leftskip2.45cm\relax \itshape}{}
{\bfseries}{}{0pt}{%
\makebox[0pt][r]{%
  \smash{\parbox[t]{3.24cm}{\raggedright\thmname{#1}%
  \thmnumber{\space #2}\thmnote{\newline (#3)}}}%
  \hspace{.5cm}}}
\theoremstyle{rq}
\newtheorem{remarque}{Remark}[chapter]


%%%%%%%%%%%%%%% Définition
\newtheoremstyle{Marine2}% name 
{}% Space above 
{1cm}% Space below 
{\advance \leftskip2.45cm\relax \itshape}% Body font 
{}% Indent amount 
{\bfseries}{}{0pt}
{\makebox[0pt][r]{
  \smash{\parbox[t]{3.25cm}
	    {\raggedright\thmname{#1}%
         \thmnumber{\space #2}\thmnote{\newline 
         \textcolor{BrickRed}{ (#3)}}}}
  	 \hspace{.5cm}}}


\theoremstyle{Marine2}
\newtheorem{definition}{Definition}[section]













\newtcolorbox[auto counter, number within=section]{theoreme}[2][]{%
    colback=white!95!roug,
    colframe=roug,
    colbacktitle=white!80!roug,
    coltitle=black,
    fonttitle=\bfseries, 
    title=Theorem~\thetcbcounter.\ #2,
    enhanced,
    before={\vspace{0.4cm}}, 
    after={\vspace{0.1cm}},
    attach boxed title to top left={yshift=-2mm, xshift=0.4cm},%
    #1% For possible options
}



\tcbset{   
   %%%%%%%%%%%%%%%%%%%%%%%%%%%%%%%%%%%%%%%%%%% EXEMPLES
    SV/.style={
    enhanced,
        breakable,
        sharp corners=all,
        fonttitle=\bfseries\normalsize,
        fontupper=\normalsize\itshape,
        colframe=white,
        before={\vspace{0.2cm}}, 
        after={\vspace{0.2cm}},      
        attach boxed title to top left={xshift=-.232\linewidth, yshift= -7 mm},
        minipage boxed title=.17\linewidth,
        left skip={.12\linewidth},
    coltitle=vert, colback=white!95!vert, colbacktitle=white,
        overlay unbroken ={
            \draw[vert][thick] (title.south west)--(title.south east);
            \draw[vert][thick] ([xshift=3.5mm]frame.north west)--([xshift=3.5mm]frame.south west);},
        overlay first={
            \draw[vert][thick] (title.south west)--(title.south east); 
            \draw[vert][thick] ([xshift=3.5mm]frame.north west)--([xshift=3.5mm]frame.south west);},
        overlay middle={
            \draw[vert][thick] ([xshift=3.5mm]frame.north west)--([xshift=3.5mm]frame.south west);},
        overlay last={
            \draw[vert][thick] ([xshift=3.5mm]frame.north west)--([xshift=3.5mm]frame.south west);},
        },
        %%%%%%%%%%%%%%%%%%%%%%%%%%%%%%%%%%% EXERCICES DE COMPREHENSION
    SO/.style={
    enhanced,
        breakable,
        sharp corners=all,
        fonttitle=\bfseries\normalsize,
        fontupper=\normalsize\itshape,
        colframe=white,
        before={\vspace{0.1cm}}, 
        after={\vspace{0.3cm}},      
         attach boxed title to top left,
        boxed title style={empty, size=minimal, bottom=1.5mm},
    coltitle=oran, colback=white!95!oran, colbacktitle=white,
        overlay unbroken ={
            \draw[oran][thick] (title.south west)--(title.south east);
            \draw[oran][thick] ([xshift=3.5mm]frame.north west)--([xshift=3.5mm]frame.south west);},
        overlay first={
            \draw[oran][thick] (title.south west)--(title.south east); 
            \draw[oran][thick] ([xshift=3.5mm]frame.north west)--([xshift=3.5mm]frame.south west);},
        overlay middle={
            \draw[oran][thick] ([xshift=3.5mm]frame.north west)--([xshift=3.5mm]frame.south west);},
        overlay last={
            \draw[oran][thick] ([xshift=3.5mm]frame.north west)--([xshift=3.5mm]frame.south west);},
        },   
                %%%%%%%%%%%%%%%%%%%%%%%%%%%%%%%%%%% EXERCICES DE ANNONCES
    SO*/.style={
    enhanced,
        breakable,
        sharp corners=all,
        fonttitle=\bfseries\normalsize,
        fontupper=\normalsize\itshape,
        colframe=white,
        before={\vspace{0.2cm}}, 
        after={\vspace{0.2cm}},      
        attach boxed title to top left={xshift=-.232\linewidth, yshift= -7 mm},
        minipage boxed title=.17\linewidth,
        left skip={.12\linewidth},
    coltitle=oran, colback=white!95!oran, colbacktitle=white,
        overlay unbroken ={
            \draw[oran][thick] (title.south west)--(title.south east);
            \draw[oran][thick] ([xshift=3.5mm]frame.north west)--([xshift=3.5mm]frame.south west);},
        overlay first={
            \draw[oran][thick] (title.south west)--(title.south east); 
            \draw[oran][thick] ([xshift=3.5mm]frame.north west)--([xshift=3.5mm]frame.south west);},
        overlay middle={
            \draw[oran][thick] ([xshift=3.5mm]frame.north west)--([xshift=3.5mm]frame.south west);},
        overlay last={
            \draw[oran][thick] ([xshift=3.5mm]frame.north west)--([xshift=3.5mm]frame.south west);},
        },   
        %%%%%%%%%%%%%%%%%%%%%%%%%%%%%%%%%%%%%%%% DEMONSTRATION
    SQ/.style={
        enhanced,
        breakable,
        sharp corners=all,
        fonttitle=\bfseries\normalsize,
        fontupper=\normalsize\itshape,
        colframe=white,
        colback=white!95!viol, 
        colbacktitle=white,
        coltitle=viol, 
        before={\vspace{0.1cm}}, 
        after={\vspace{0.8cm}},      
        attach boxed title to top left={xshift=-.23\linewidth, yshift= -11 mm}, % xshift, how much you shift title in comparison to the box
        minipage boxed title=.155\linewidth,
        after upper={\hfill$\qed$},
        left skip={.19\linewidth}, % this changes how far you put the title
        overlay unbroken ={
            \draw[viol][thick] (title.south west)--(title.south east);
            \draw[viol][thick] ([xshift=3.5mm]frame.north west)|-([xshift=15mm]frame.south west);},
        overlay first={
            \draw[viol][thick] (title.south west)--(title.south east); 
            \draw[viol][thick] ([xshift=3.5mm]frame.north west)--([xshift=3.5mm]frame.south west);},
        overlay middle={
            \draw[viol][thick] ([xshift=3.5mm]frame.north west)--([xshift=3.5mm]frame.south west);},
        overlay last={
            \draw[viol][thick] ([xshift=3.5mm]frame.north west)|-([xshift=15mm]frame.south west);},
    },
    %%%%%%%%%%%%%%%%%%%%%%%%%%%%%%%%%%%%%%% EXERCICES DENTRAINEMENT 
    ST/.style={
    enhanced,
        breakable,
        sharp corners=all,
        fonttitle=\bfseries\normalsize,
        fontupper=\normalsize\itshape,
        colframe=white,
        before={\vspace{0.1cm}}, 
        after={\vspace{0.3cm}},      
        attach boxed title to top left,
        boxed title style={empty, size=minimal, bottom=1.5mm},
    coltitle=oran, colback=white, colbacktitle=white,
        overlay unbroken ={
            \draw[oran][thick] (title.south west)--(title.south east);
            \draw[oran][thick] ([xshift=3.5mm]frame.north west)--([xshift=3.5mm]frame.south west);},
        overlay first={
            \draw[oran][thick] (title.south west)--(title.south east); 
            \draw[oran][thick] ([xshift=3.5mm]frame.north west)--([xshift=3.5mm]frame.south west);},
        overlay middle={
            \draw[oran][thick] ([xshift=3.5mm]frame.north west)--([xshift=3.5mm]frame.south west);},
        overlay last={
            \draw[oran][thick] ([xshift=3.5mm]frame.north west)--([xshift=3.5mm]frame.south west);},
        },   
        %%%%%%%%%%%%%%%%%%%%%%%%%%%%%%%%%%%%%%%%%%% SOLUTIONS
    SS/.style={
    enhanced,
        breakable,
        sharp corners=all,
        fonttitle=\bfseries\normalsize,
        fontupper=\normalsize\itshape,
        colframe=white,
        before={\vspace{0.1cm}}, 
        after={\vspace{0.3cm}}, 
        attach boxed title to top left,
        boxed title style={empty, size=minimal, bottom=1.5mm},     
        after upper={\hfill$\qed$},
    coltitle=mauv, colback=white!95!mauv, colbacktitle=white,
        overlay unbroken ={
            \draw[mauv][thick] (title.south west)--(title.south east);
            \draw[mauv][thick] ([xshift=3.5mm]frame.north west)|-([xshift=15mm]frame.south west);
        },
        overlay first={
            \draw[mauv][thick] (title.south west)--(title.south east); 
            \draw[mauv][thick] ([xshift=3.5mm]frame.north west)--([xshift=3.5mm]frame.south west);},
        overlay middle={
            \draw[mauv][thick] ([xshift=3.5mm]frame.north west)--([xshift=3.5mm]frame.south west);},
        overlay last={
            \draw[mauv][thick] ([xshift=3.5mm]frame.north west)|-([xshift=15mm]frame.south west);
            },
    },  
        %%%%%%%%%%%%%%%%%%%%%%%%%%%%%%%%%%%%%%%%%%% SOLUTIONS
    SSE/.style={
    enhanced,
        breakable,
        sharp corners=all,
        fonttitle=\bfseries\normalsize,
        fontupper=\normalsize\itshape,
        colframe=white,
        before={\vspace{0.1cm}}, 
        after={\vspace{0.3cm}},      
 attach boxed title to top left,
        boxed title style={empty, size=minimal, bottom=1.5mm},
    coltitle=mauv, colback=white, colbacktitle=white,
        overlay unbroken ={
            \draw[mauv][thick] (title.south west)--(title.south east);
            \draw[mauv][thick] ([xshift=3.5mm]frame.north west)|-([xshift=15mm]frame.south west);
            },
        overlay first={
            \draw[mauv][thick] (title.south west)--(title.south east); 
            \draw[mauv][thick] ([xshift=3.5mm]frame.north west)--([xshift=3.5mm]frame.south west);},
        overlay middle={
            \draw[mauv][thick] ([xshift=3.5mm]frame.north west)--([xshift=3.5mm]frame.south west);},
        overlay last={
            \draw[mauv][thick] ([xshift=3.5mm]frame.north west)|-([xshift=15mm]frame.south west);
            },
    },  
}





\newtcbtheorem[auto counter, number within=section]{ajoutationV}{Example}{SV}{theo}
\newtcbtheorem[auto counter, number within=section]{demo}{Proof}{SQ}{theo}

%---------------------------------

\newtcolorbox[auto counter, number within=chapter]{entrainement}[2][]{title={\hypertarget{exer:#2}{Entrainement \thetcbcounter.\hfill\hyperlink{sol:#2}{Solution}}},#1, ST}

\newtcolorbox[auto counter, number within=chapter]{soluENT}[2][]{title={\hypertarget{sol:#2}{Correction \thetcbcounter.\hfill\hyperlink{exer:#2}{Exercise}}},#1, SSE}

%---------------------------------

\newtcolorbox[use counter from=entrainement, number within=chapter]{ajoutationO}[2][]{title={\hypertarget{exer:#2}{Exercice \thetcbcounter.\hfill\hyperlink{sol:#2}{Solution}}},#1, SO}

\newtcolorbox[use counter from=soluENT, number within=chapter]{solu}[2][]{title={\hypertarget{sol:#2}{Solution \thetcbcounter.\hfill\hyperlink{exer:#2}{Exercice}}},#1, SS}



%annonce d'exos

\newtcbtheorem[auto counter, number within=chapter]{ajoutationO*}{Exercices}{SO*}{theo}


%%%%%%%%%%%%%% change fontsize and properties of titles	%%%%%%%%%%%%%%%%%%%%%%%%%%%%%%%%%%

\usepackage{titlesec} %pour redéfinir les headers

\newcommand{\sectionbreak}{\clearpage}

\titleformat{\section}
{\Large\bfseries\color[RGB]{10, 80, 144}}
{\textcolor[RGB]{10, 80, 144}{~ \thesection}}
{1em}{}

\titleformat{\subsection}
{\large\bfseries}
{\rlap{\color[RGB]{238,243,252}\rule[-1.25ex]{\textwidth}{4ex}}\textcolor{Black}{~ \thesubsection}}
{1em}{}
% Alice blue (240, 248, 255)

\titleformat*{\subsubsection}{\large\bfseries}
\titleformat*{\paragraph}{\large\bfseries}
\titleformat*{\subparagraph}{\large\bfseries}


\titlespacing*{\subsection}
{0pt}{10mm}{7mm}







%%%%%%%%%%%%%%%%%%%style front%%%%%%%%%%%%%%%%%%%%%%%%%%%%%%%%%%%%%%%%%	
	\fancypagestyle{front}{%
  		\fancyhf{}%on vide les entêtes
  		\fancyfoot[C]{page \thepage}%
  		\renewcommand{\headrulewidth}{0pt}%trait horizontal pour l'entête
  		\renewcommand{\footrulewidth}{0.4pt}%trait horizontal pour les pieds de pages
  		}


%%%%%%%%%%%%%%%%%%%style main%%%%%%%%%%%%%%%%%%%%%%%%%%%%%%%%%%%%
	\fancypagestyle{main}{%
		\fancyhf{}
  		\renewcommand{\chaptermark}[1]{\markboth{\chaptername\ \thechapter.\ ##1}{}}% redéfintion pour avoir ici les titres des chapitres des sections en minuscules
  		\renewcommand{\sectionmark}[1]{\markright{\thesection\ ##1}}
		\fancyhead[c]{}
		\fancyhead[RO,LE]{\rightmark}%
  		\fancyhead[LO,RE]{\leftmark}
		\fancyfoot[C]{}
		\fancyfoot[RO,LE]{page \thepage}%
  		\fancyfoot[LO,RE]{contact : niels.carioukotlarek@yahoo.com}
  		}


%%%%%%%%%%%%%%%%%%%style back%%%%%%%%%%%%%%%%%%%%%%%%%%%%%%%%%%%%%%%%%	
	\fancypagestyle{back}{%
  		\fancyhf{}%on vide les entêtes
  		\fancyfoot[C]{page \thepage}%
  		\renewcommand{\headrulewidth}{0pt}%trait horizontal pour l'entête
  		\renewcommand{\footrulewidth}{0.4pt}%trait horizontal pour les pieds de pages
		}






%%%%%=======================================================================tikz
\usepackage{tikz}
\usetikzlibrary{arrows.meta}
\usepackage{mathdots}
\usepackage{mathtools}
\usepackage{pdflscape}
\usepackage{pgfplots}
\usepackage{siunitx}
\usepackage{slashed}
\usepackage{tabularx}
\usepackage{tikz}
\usepackage{tkz-euclide}
\usepackage[normalem]{ulem}
\usepackage[all]{xy}
\usepackage{imakeidx}



\pgfarrowsdeclarecombine{twolatex'}{twolatex'}{latex'}{latex'}{latex'}{latex'}
\tikzset{->>/.style = {decoration={markings,
                                  mark=at position 1 with {\arrow[scale=2]{latex'}}},
                      postaction={decorate}}}
\tikzset{<-/.style = {decoration={markings,
                                  mark=at position 0 with {\arrowreversed[scale=2]{latex'}}},
                      postaction={decorate}}}
\tikzset{<->/.style = {decoration={markings,
                                   mark=at position 0 with {\arrowreversed[scale=2]{latex'}},
                                   mark=at position 1 with {\arrow[scale=2]{latex'}}},
                       postaction={decorate}}}
\tikzset{->-/.style = {decoration={markings,
                                   mark=at position #1 with {\arrow[scale=2]{latex'}}},
                       postaction={decorate}}}
\tikzset{-<-/.style = {decoration={markings,
                                   mark=at position #1 with {\arrowreversed[scale=2]{latex'}}},
                       postaction={decorate}}}
\tikzset{<<-/.style = {decoration={markings,
                                  mark=at position 0 with {\arrowreversed[scale=2]{twolatex'}}},
                      postaction={decorate}}}
\tikzset{<<->>/.style = {decoration={markings,
                                   mark=at position 0 with {\arrowreversed[scale=2]{twolatex'}},
                                   mark=at position 1 with {\arrow[scale=2]{twolatex'}}},
                       postaction={decorate}}}
\tikzset{->>-/.style = {decoration={markings,
                                   mark=at position #1 with {\arrow[scale=2]{twolatex'}}},
                       postaction={decorate}}}
\tikzset{-<<-/.style = {decoration={markings,
                                   mark=at position #1 with {\arrowreversed[scale=2]{twolatex'}}},
                       postaction={decorate}}}

\tikzset{circ/.style = {fill, circle, inner sep = 0, minimum size = 3}}
\tikzset{scirc/.style = {fill, circle, inner sep = 0, minimum size = 1.5}}
\tikzset{mstate/.style={circle, draw, blue, text=black, minimum width=0.7cm}}

\tikzset{eqpic/.style={baseline={([yshift=-.5ex]current bounding box.center)}}}
\tikzset{commutative diagrams/.cd,cdmap/.style={/tikz/column 1/.append style={anchor=base east},/tikz/column 2/.append style={anchor=base west},row sep=tiny}}

\definecolor{mblue}{rgb}{0.17, 0.27, 0.9}
\definecolor{morange}{rgb}{1, 0.48, 0}
\definecolor{mgreen}{rgb}{0, 0.4, 0.1}
\definecolor{mred}{rgb}{0.8, 0.09, 0.09}







%%% définition de couleurs pour théoremes
\definecolor{vert}{RGB}{0,181,0}
\definecolor{oran}{RGB}{223,74,0}
\definecolor{viol}{RGB}{134,0,175}
\definecolor{mauv}{RGB}{200,0,115}
\definecolor{roug}{RGB}{215,15,0}	

%%%% couleurs listes :
\definecolor{abricot}{rgb}{0.901,0.494,0.188}
\definecolor{absinthe}{rgb}{0.498,0.866,0.298}
\definecolor{acajou}{rgb}{0.533,0.258,0.113}
\definecolor{aiguemarine}{rgb}{0.474,0.972,0.972}
\definecolor{ailedecorbeau}{rgb}{0.0,0.0,0.0}
\definecolor{albatre}{rgb}{0.996,0.996,0.996}
\definecolor{alezan}{rgb}{0.654,0.403,0.149}
\definecolor{amande}{rgb}{0.509,0.768,0.423}
\definecolor{amarante}{rgb}{0.568,0.156,0.231}
\definecolor{ambrejaune}{rgb}{0.941,0.764,0.0}
\definecolor{ambrerouge}{rgb}{0.678,0.223,0.054}
\definecolor{amethyste}{rgb}{0.533,0.301,0.654}
\definecolor{anthracite}{rgb}{0.188,0.188,0.188}
\definecolor{aquilain}{rgb}{0.678,0.309,0.035}
\definecolor{ardoise}{rgb}{0.352,0.368,0.419}
\definecolor{argent}{rgb}{0.807,0.807,0.807}
\definecolor{argile}{rgb}{0.937,0.937,0.937}
\definecolor{asperge}{rgb}{0.482,0.627,0.356}
\definecolor{aubergine}{rgb}{0.215,0.0,0.156}
\definecolor{auburn}{rgb}{0.615,0.243,0.047}
\definecolor{aurore}{rgb}{1.0,0.796,0.376}
\definecolor{avocat}{rgb}{0.337,0.509,0.011}
\definecolor{azur}{rgb}{0.0,0.498,1.0}
\definecolor{azurbrume}{rgb}{0.941,1.0,1.0}
\definecolor{azurclair}{rgb}{0.454,0.815,0.945}
\definecolor{azurin}{rgb}{0.662,0.917,0.996}
\definecolor{baillet}{rgb}{0.682,0.392,0.176}
\definecolor{banane}{rgb}{0.819,0.713,0.023}
\definecolor{basane}{rgb}{0.545,0.423,0.258}
\definecolor{beige}{rgb}{0.784,0.678,0.498}
\definecolor{beigeclair}{rgb}{0.96,0.96,0.862}
\definecolor{beigeasse}{rgb}{0.686,0.654,0.482}
\definecolor{beurre}{rgb}{0.941,0.89,0.419}
\definecolor{beurrefrais}{rgb}{1.0,0.956,0.552}
\definecolor{bis}{rgb}{0.462,0.435,0.392}
\definecolor{bisque}{rgb}{1.0,0.894,0.768}
\definecolor{bistre}{rgb}{0.521,0.427,0.301}
\definecolor{bitume}{rgb}{0.305,0.239,0.156}
\definecolor{blanc}{rgb}{1.0,1.0,1.0}
\definecolor{blanccasse}{rgb}{0.996,0.996,0.886}
\definecolor{blanccreme}{rgb}{0.992,0.945,0.721}
\definecolor{blancdargent}{rgb}{0.996,0.996,0.996}
\definecolor{blancdespagne}{rgb}{0.996,0.992,0.941}
\definecolor{blancdelait}{rgb}{0.984,0.988,0.98}
\definecolor{blancdemeudon}{rgb}{0.996,0.992,0.941}
\definecolor{blancdezinc}{rgb}{0.964,0.996,0.996}
\definecolor{blanclunaire}{rgb}{0.956,0.996,0.996}
\definecolor{ble}{rgb}{0.909,0.839,0.188}
\definecolor{blet}{rgb}{0.356,0.235,0.066}
\definecolor{bleu}{rgb}{0.105,0.003,0.607}
\definecolor{bleuacier}{rgb}{0.227,0.556,0.729}
\definecolor{bleuardoise}{rgb}{0.407,0.435,0.549}
\definecolor{bleubarbeau}{rgb}{0.329,0.447,0.682}
\definecolor{bleucanard}{rgb}{0.015,0.545,0.603}
\definecolor{bleuceleste}{rgb}{0.149,0.768,0.925}
\definecolor{bleuceruleen}{rgb}{0.207,0.478,0.717}
\definecolor{bleucharrette}{rgb}{0.556,0.635,0.776}
\definecolor{bleuciel}{rgb}{0.466,0.709,0.996}
\definecolor{bleudeberlin}{rgb}{0.141,0.266,0.36}
\definecolor{bleudecobalt}{rgb}{0.133,0.258,0.486}
\definecolor{bleudefrance}{rgb}{0.192,0.549,0.905}
\definecolor{bleudeminuit}{rgb}{0.0,0.2,0.4}
\definecolor{bleudeprusse}{rgb}{0.141,0.266,0.36}
\definecolor{bleudesmersdusud}{rgb}{0.0,0.8,0.796}
\definecolor{bleudragee}{rgb}{0.874,0.949,1.0}
\definecolor{bleuelectrique}{rgb}{0.172,0.458,1.0}
\definecolor{bleufumee}{rgb}{0.733,0.823,0.882}
\definecolor{bleugivre}{rgb}{0.501,0.815,0.815}
\definecolor{bleuguede}{rgb}{0.337,0.45,0.603}
\definecolor{bleuhussard}{rgb}{0.141,0.266,0.36}
\definecolor{bleuklein}{rgb}{0.0,0.184,0.654}
\definecolor{bleulavande}{rgb}{0.588,0.513,0.925}
\definecolor{bleumajorelle}{rgb}{0.376,0.313,0.862}
\definecolor{bleumarine}{rgb}{0.011,0.133,0.298}
\definecolor{bleunuit}{rgb}{0.058,0.019,0.419}
\definecolor{bleuoutremer}{rgb}{0.106,0.004,0.608}
\definecolor{bleupaon}{rgb}{0.023,0.466,0.564}
\definecolor{bleupersan}{rgb}{0.4,0.0,1.0}
\definecolor{bleupetrole}{rgb}{0.113,0.282,0.317}
\definecolor{bleuprimaire}{rgb}{0.0,0.0,1.0}
\definecolor{bleuroi}{rgb}{0.192,0.549,0.905}
\definecolor{bleusaphir}{rgb}{0.003,0.192,0.705}
\definecolor{bleuturquin}{rgb}{0.258,0.356,0.541}
\definecolor{bleuturquoise}{rgb}{0.145,0.992,0.913}
\definecolor{blond}{rgb}{0.886,0.737,0.454}
\definecolor{blondvenitien}{rgb}{0.905,0.658,0.329}
\definecolor{bordeaux}{rgb}{0.427,0.027,0.101}
\definecolor{bourgogne}{rgb}{0.419,0.05,0.05}
\definecolor{boutondor}{rgb}{0.988,0.862,0.07}
\definecolor{brique}{rgb}{0.517,0.18,0.105}
\definecolor{bronze}{rgb}{0.38,0.305,0.101}
\definecolor{broudenoix}{rgb}{0.247,0.133,0.015}
\definecolor{brun}{rgb}{0.356,0.235,0.066}
\definecolor{brunclair}{rgb}{0.803,0.521,0.247}
\definecolor{bureau}{rgb}{0.419,0.341,0.192}
\definecolor{byzantin}{rgb}{0.741,0.2,0.643}
\definecolor{byzantium}{rgb}{0.439,0.16,0.388}
\definecolor{cacadoie}{rgb}{0.803,0.803,0.05}
\definecolor{cacao}{rgb}{0.38,0.294,0.227}
\definecolor{cachou}{rgb}{0.184,0.105,0.047}
\definecolor{cafe}{rgb}{0.274,0.18,0.003}
\definecolor{cafeaulait}{rgb}{0.47,0.368,0.184}
\definecolor{cannelle}{rgb}{0.494,0.345,0.207}
\definecolor{capucine}{rgb}{1.0,0.368,0.301}
\definecolor{caramel}{rgb}{0.494,0.2,0.0}
\definecolor{carmin}{rgb}{0.588,0.0,0.094}
\definecolor{carotte}{rgb}{0.956,0.4,0.105}
\definecolor{cassis}{rgb}{0.227,0.007,0.05}
\definecolor{celadon}{rgb}{0.513,0.65,0.592}
\definecolor{cerise}{rgb}{0.87,0.192,0.388}
\definecolor{ceruse}{rgb}{0.996,0.996,0.996}
\definecolor{chair}{rgb}{0.996,0.764,0.674}
\definecolor{chamois}{rgb}{0.815,0.752,0.478}
\definecolor{champagne}{rgb}{0.984,0.949,0.717}
\definecolor{charbonneux}{rgb}{0.0,0.0,0.062}
\definecolor{chartreuse}{rgb}{0.498,1.0,0.0}
\definecolor{chataigne}{rgb}{0.501,0.427,0.352}
\definecolor{chatain}{rgb}{0.545,0.423,0.258}
\definecolor{chaudron}{rgb}{0.521,0.325,0.058}
\definecolor{chenu}{rgb}{0.996,0.996,0.996}
\definecolor{chocolat}{rgb}{0.352,0.227,0.133}
\definecolor{cinabre}{rgb}{0.858,0.09,0.007}
\definecolor{citron}{rgb}{0.968,1.0,0.235}
\definecolor{citrouille}{rgb}{0.874,0.427,0.078}
\definecolor{clarissimo}{rgb}{0.725,0.698,0.462}
\definecolor{claro}{rgb}{0.517,0.352,0.231}
\definecolor{claroclaro}{rgb}{0.729,0.607,0.38}
\definecolor{colombin}{rgb}{0.415,0.27,0.364}
\definecolor{colorado}{rgb}{0.439,0.207,0.086}
\definecolor{coloradoclaro}{rgb}{0.415,0.294,0.129}
\definecolor{coquelicot}{rgb}{0.776,0.031,0.0}
\definecolor{coquilledoeuf}{rgb}{0.992,0.913,0.878}
\definecolor{corail}{rgb}{0.905,0.243,0.003}
\definecolor{cramoisi}{rgb}{0.862,0.078,0.235}
\definecolor{creme}{rgb}{0.992,0.945,0.721}
\definecolor{cuissedenymphe}{rgb}{0.996,0.905,0.941}
\definecolor{cuissedenympheemue}{rgb}{1.0,0.411,0.705}
\definecolor{cuivre}{rgb}{0.701,0.403,0.0}
\definecolor{cyan}{rgb}{0.168,0.98,0.98}
\definecolor{cyansecondaire}{rgb}{0.0,1.0,1.0}
\definecolor{ceruleum}{rgb}{0.207,0.478,0.717}
\definecolor{denim}{rgb}{0.082,0.376,0.741}
\definecolor{dium}{rgb}{0.043,0.086,0.086}
\definecolor{ebene}{rgb}{0.0,0.0,0.0}
\definecolor{ecarlate}{rgb}{0.929,0.0,0.0}
\definecolor{ecru}{rgb}{0.996,0.996,0.878}
\definecolor{emeraude}{rgb}{0.003,0.843,0.345}
\definecolor{etainoxyde}{rgb}{0.729,0.729,0.729}
\definecolor{etainpur}{rgb}{0.929,0.929,0.929}
\definecolor{fauve}{rgb}{0.678,0.309,0.035}
\definecolor{fer}{rgb}{0.517,0.517,0.517}
\definecolor{feuvif}{rgb}{1.0,0.286,0.003}
\definecolor{feuillemorte}{rgb}{0.6,0.317,0.168}
\definecolor{flave}{rgb}{0.901,0.901,0.592}
\definecolor{fleurdesoufre}{rgb}{1.0,1.0,0.419}
\definecolor{fraise}{rgb}{0.749,0.188,0.188}
\definecolor{fraiseecrasee}{rgb}{0.643,0.141,0.141}
\definecolor{framboise}{rgb}{0.78,0.172,0.282}
\definecolor{fuchsia}{rgb}{0.956,0.0,0.631}
\definecolor{garance}{rgb}{0.933,0.062,0.062}
\definecolor{glauque}{rgb}{0.392,0.607,0.533}
\definecolor{glycine}{rgb}{0.788,0.627,0.862}
\definecolor{grege}{rgb}{0.733,0.682,0.596}
\definecolor{grenadine}{rgb}{0.913,0.219,0.247}
\definecolor{grenat}{rgb}{0.431,0.043,0.078}
\definecolor{gris}{rgb}{0.376,0.376,0.376}
\definecolor{grisacier}{rgb}{0.686,0.686,0.686}
\definecolor{grisanthracite}{rgb}{0.188,0.188,0.188}
\definecolor{grisdelin}{rgb}{0.823,0.792,0.925}
\definecolor{grisdemaure}{rgb}{0.407,0.368,0.262}
\definecolor{grisdepayne}{rgb}{0.403,0.443,0.474}
\definecolor{grisfer}{rgb}{0.498,0.498,0.498}
\definecolor{grisfumee}{rgb}{0.733,0.823,0.882}
\definecolor{grisperle}{rgb}{0.807,0.807,0.807}
\definecolor{grisplomb}{rgb}{0.474,0.501,0.505}
\definecolor{grissouris}{rgb}{0.619,0.619,0.619}
\definecolor{gristaupe}{rgb}{0.274,0.247,0.196}
\definecolor{gristourdille}{rgb}{0.756,0.749,0.694}
\definecolor{gristourterelle}{rgb}{0.733,0.674,0.674}
\definecolor{groseille}{rgb}{0.811,0.039,0.113}
\definecolor{havane}{rgb}{0.58,0.498,0.376}
\definecolor{heliotrope}{rgb}{0.874,0.45,1.0}
\definecolor{hoto}{rgb}{0.0,0.0,0.0}
\definecolor{incarnadin}{rgb}{0.996,0.588,0.627}
\definecolor{incarnat}{rgb}{1.0,0.435,0.49}
\definecolor{indigo}{rgb}{0.18,0.0,0.423}
\definecolor{indigochaud}{rgb}{0.474,0.109,0.972}
\definecolor{indigoduweb}{rgb}{0.294,0.0,0.509}
\definecolor{indigoelectrique}{rgb}{0.435,0.0,1.0}
\definecolor{isabelle}{rgb}{0.47,0.368,0.184}
\definecolor{ivoire}{rgb}{1.0,1.0,0.831}
\definecolor{jade}{rgb}{0.529,0.913,0.564}
\definecolor{jais}{rgb}{0.0,0.0,0.0}
\definecolor{jauneaureolin}{rgb}{0.992,0.933,0.0}
\definecolor{jaunebanane}{rgb}{0.819,0.713,0.023}
\definecolor{jauneboutondor}{rgb}{0.964,0.862,0.07}
\definecolor{jaunecanari}{rgb}{0.905,0.941,0.05}
\definecolor{jaunechartreuse}{rgb}{0.874,1.0,0.0}
\definecolor{jaunecitron}{rgb}{0.968,1.0,0.235}
\definecolor{jaunedor}{rgb}{0.937,0.847,0.027}
\definecolor{jaunedechrome}{rgb}{0.929,1.0,0.047}
\definecolor{jaunedemars}{rgb}{0.933,0.819,0.325}
\definecolor{jaunedenaples}{rgb}{1.0,0.941,0.737}
\definecolor{jaunefleurdesoufre}{rgb}{1.0,1.0,0.419}
\definecolor{jauneimperial}{rgb}{1.0,0.894,0.211}
\definecolor{jaunemais}{rgb}{1.0,0.87,0.458}
\definecolor{jaunemimosa}{rgb}{0.996,0.972,0.423}
\definecolor{jaunemoutarde}{rgb}{0.78,0.811,0.0}
\definecolor{jaunenankin}{rgb}{0.968,0.886,0.411}
\definecolor{jaunepaille}{rgb}{0.996,0.89,0.278}
\definecolor{jaunepoussin}{rgb}{0.968,0.89,0.372}
\definecolor{jauneprimaire}{rgb}{1.0,1.0,0.0}
\definecolor{jaunesoufre}{rgb}{1.0,1.0,0.419}
\definecolor{kaki}{rgb}{0.58,0.505,0.168}
\definecolor{lapislazuli}{rgb}{0.149,0.38,0.611}
\definecolor{lavalliere}{rgb}{0.56,0.349,0.133}
\definecolor{lavande}{rgb}{0.588,0.513,0.925}
\definecolor{liedevin}{rgb}{0.674,0.117,0.266}
\definecolor{lilas}{rgb}{0.713,0.4,0.823}
\definecolor{lin}{rgb}{0.98,0.941,0.901}
\definecolor{maduro}{rgb}{0.215,0.184,0.145}
\definecolor{madurocolorado}{rgb}{0.356,0.235,0.066}
\definecolor{magentafonce}{rgb}{0.501,0.0,0.501}
\definecolor{magentafushia}{rgb}{0.858,0.0,0.45}
\definecolor{magentasecondaire}{rgb}{1.0,0.0,1.0}
\definecolor{mais}{rgb}{1.0,0.87,0.458}
\definecolor{malachite}{rgb}{0.121,0.627,0.333}
\definecolor{mandarine}{rgb}{0.996,0.639,0.278}
\definecolor{marine}{rgb}{0.011,0.133,0.298}
\definecolor{marron}{rgb}{0.345,0.16,0.0}
\definecolor{mastic}{rgb}{0.701,0.694,0.568}
\definecolor{mauve}{rgb}{0.831,0.45,0.831}
\definecolor{melon}{rgb}{0.87,0.596,0.086}
\definecolor{menthe}{rgb}{0.086,0.721,0.305}
\definecolor{menthealeau}{rgb}{0.329,0.976,0.552}
\definecolor{miel}{rgb}{0.854,0.701,0.039}
\definecolor{mordore}{rgb}{0.529,0.349,0.101}
\definecolor{moreau}{rgb}{0.0,0.0,0.0}
\definecolor{moutarde}{rgb}{0.78,0.811,0.0}
\definecolor{nacarat}{rgb}{0.988,0.364,0.364}
\definecolor{nankin}{rgb}{0.968,0.886,0.411}
\definecolor{neige}{rgb}{0.996,0.996,0.996}
\definecolor{noir}{rgb}{0.0,0.0,0.0}
\definecolor{noiranimal}{rgb}{0.0,0.0,0.0}
\definecolor{noircharbon}{rgb}{0.0,0.0,0.062}
\definecolor{noirdaniline}{rgb}{0.07,0.05,0.086}
\definecolor{noirdencre}{rgb}{0.0,0.0,0.0}
\definecolor{noirdivoire}{rgb}{0.0,0.0,0.0}
\definecolor{noirdecarbone}{rgb}{0.074,0.054,0.039}
\definecolor{noirdefumee}{rgb}{0.074,0.054,0.039}
\definecolor{noirdejais}{rgb}{0.0,0.0,0.0}
\definecolor{noiraud}{rgb}{0.184,0.117,0.054}
\definecolor{noisette}{rgb}{0.584,0.337,0.156}
\definecolor{ocrejaune}{rgb}{0.874,0.686,0.172}
\definecolor{ocrerouge}{rgb}{0.866,0.596,0.36}
\definecolor{olive}{rgb}{0.439,0.552,0.137}
\definecolor{opalin}{rgb}{0.949,1.0,1.0}
\definecolor{or}{rgb}{1.0,0.843,0.0}
\definecolor{orange}{rgb}{0.929,0.498,0.062}
\definecolor{orangebrulee}{rgb}{0.8,0.333,0.0}
\definecolor{orchidee}{rgb}{0.854,0.439,0.839}
\definecolor{orpiment}{rgb}{0.988,0.823,0.109}
\definecolor{orpindeperse}{rgb}{0.988,0.823,0.109}
\definecolor{oscuro}{rgb}{0.16,0.129,0.027}
\definecolor{paille}{rgb}{0.996,0.89,0.278}
\definecolor{papaye}{rgb}{1.0,0.937,0.835}
\definecolor{papierbulle}{rgb}{0.929,0.827,0.549}
\definecolor{parme}{rgb}{0.811,0.627,0.913}
\definecolor{passevelours}{rgb}{0.568,0.156,0.231}
\definecolor{pastel}{rgb}{0.337,0.45,0.603}
\definecolor{peche}{rgb}{0.992,0.749,0.717}
\definecolor{peluredoignon}{rgb}{0.835,0.517,0.564}
\definecolor{pervenche}{rgb}{0.8,0.8,1.0}
\definecolor{pinchard}{rgb}{0.8,0.8,0.8}
\definecolor{pistache}{rgb}{0.745,0.96,0.454}
\definecolor{platine}{rgb}{0.98,0.941,0.772}
\definecolor{plomb}{rgb}{0.474,0.501,0.505}
\definecolor{poildechameau}{rgb}{0.713,0.47,0.137}
\definecolor{ponceau}{rgb}{0.776,0.031,0.0}
\definecolor{prasin}{rgb}{0.298,0.65,0.419}
\definecolor{prune}{rgb}{0.505,0.078,0.325}
\definecolor{puce}{rgb}{0.305,0.086,0.035}
\definecolor{queuederenard}{rgb}{0.568,0.156,0.231}
\definecolor{queuedevacheclair}{rgb}{0.764,0.705,0.439}
\definecolor{queuedevachefonce}{rgb}{0.658,0.596,0.454}
\definecolor{reglisse}{rgb}{0.176,0.141,0.117}
\definecolor{rose}{rgb}{0.992,0.423,0.619}
\definecolor{rosebalais}{rgb}{0.768,0.411,0.56}
\definecolor{rosebonbon}{rgb}{0.976,0.258,0.619}
\definecolor{rosedragee}{rgb}{0.996,0.749,0.823}
\definecolor{rosefuchsia}{rgb}{0.992,0.247,0.572}
\definecolor{rosemountbatten}{rgb}{0.6,0.478,0.552}
\definecolor{rosethe}{rgb}{1.0,0.525,0.415}
\definecolor{rosevif}{rgb}{1.0,0.0,0.498}
\definecolor{rouge}{rgb}{1.0,0.0,0.0}
\definecolor{rougeandrinople}{rgb}{0.662,0.066,0.003}
\definecolor{rougeanglais}{rgb}{0.968,0.137,0.047}
\definecolor{rougebismarck}{rgb}{0.647,0.149,0.039}
\definecolor{rougebordeaux}{rgb}{0.427,0.027,0.101}
\definecolor{rougebourgogne}{rgb}{0.419,0.05,0.05}
\definecolor{rougecapucine}{rgb}{1.0,0.368,0.301}
\definecolor{rougecardinal}{rgb}{0.721,0.125,0.062}
\definecolor{rougecarmin}{rgb}{0.588,0.0,0.094}
\definecolor{rougecerise}{rgb}{0.733,0.043,0.043}
\definecolor{rougecinabre}{rgb}{0.858,0.09,0.007}
\definecolor{rougecoquelicot}{rgb}{0.776,0.031,0.0}
\definecolor{rougedalizarine}{rgb}{0.89,0.149,0.211}
\definecolor{rougedandrinople}{rgb}{0.662,0.066,0.003}
\definecolor{rougedaniline}{rgb}{0.929,0.0,0.0}
\definecolor{rougedefalun}{rgb}{0.501,0.094,0.094}
\definecolor{rougedemars}{rgb}{0.968,0.137,0.047}
\definecolor{rougeecrevisse}{rgb}{0.737,0.125,0.003}
\definecolor{rougefeu}{rgb}{0.996,0.105,0.0}
\definecolor{rougefraise}{rgb}{0.749,0.188,0.188}
\definecolor{rougeframboise}{rgb}{0.78,0.172,0.282}
\definecolor{rougegrenadine}{rgb}{0.913,0.219,0.247}
\definecolor{rougegrenat}{rgb}{0.431,0.043,0.078}
\definecolor{rougegroseille}{rgb}{0.811,0.039,0.113}
\definecolor{rougeponceau}{rgb}{0.776,0.031,0.0}
\definecolor{rougeprimaire}{rgb}{1.0,0.0,0.0}
\definecolor{rougesang}{rgb}{0.521,0.023,0.023}
\definecolor{rougesokai}{rgb}{0.572,0.0,0.09}
\definecolor{rougetomate}{rgb}{0.87,0.16,0.086}
\definecolor{rougetomette}{rgb}{0.682,0.29,0.203}
\definecolor{rougeturc}{rgb}{0.662,0.066,0.003}
\definecolor{rougevermillon}{rgb}{0.858,0.09,0.007}
\definecolor{rougeviolet}{rgb}{0.78,0.082,0.521}
\definecolor{rouille}{rgb}{0.596,0.341,0.09}
\definecolor{roux}{rgb}{0.678,0.309,0.035}
\definecolor{rubis}{rgb}{0.878,0.066,0.372}
\definecolor{sable}{rgb}{0.878,0.803,0.662}
\definecolor{safran}{rgb}{0.952,0.839,0.09}
\definecolor{safre}{rgb}{0.003,0.192,0.705}
\definecolor{sangdeboeuf}{rgb}{0.45,0.031,0.0}
\definecolor{saphir}{rgb}{0.003,0.192,0.705}
\definecolor{sarcelle}{rgb}{0.0,0.556,0.556}
\definecolor{saumon}{rgb}{0.972,0.556,0.333}
\definecolor{sepia}{rgb}{0.682,0.537,0.392}
\definecolor{smalt}{rgb}{0.0,0.2,0.6}
\definecolor{smaragdin}{rgb}{0.003,0.843,0.345}
\definecolor{soufre}{rgb}{1.0,1.0,0.419}
\definecolor{souris}{rgb}{0.619,0.619,0.619}
\definecolor{tabac}{rgb}{0.623,0.333,0.117}
\definecolor{tangerine}{rgb}{1.0,0.498,0.0}
\definecolor{taupe}{rgb}{0.274,0.247,0.196}
\definecolor{terredombre}{rgb}{0.572,0.427,0.152}
\definecolor{terredesienne}{rgb}{0.541,0.2,0.141}
\definecolor{terredesiennebrulee}{rgb}{0.913,0.454,0.317}
\definecolor{tilleuil}{rgb}{0.611,0.678,0.125}
\definecolor{tomate}{rgb}{0.87,0.16,0.086}
\definecolor{topaze}{rgb}{0.98,0.917,0.45}
\definecolor{tourterelle}{rgb}{0.733,0.674,0.674}
\definecolor{turquoise}{rgb}{0.145,0.992,0.913}
\definecolor{vanille}{rgb}{0.882,0.807,0.603}
\definecolor{ventredebiche}{rgb}{0.913,0.788,0.694}
\definecolor{vermeil}{rgb}{1.0,0.035,0.129}
\definecolor{vermillon}{rgb}{0.858,0.09,0.007}
\definecolor{vertabsinthe}{rgb}{0.498,0.866,0.298}
\definecolor{vertamande}{rgb}{0.509,0.768,0.423}
\definecolor{vertanis}{rgb}{0.623,0.909,0.333}
\definecolor{vertavocat}{rgb}{0.337,0.509,0.011}
\definecolor{vertbouteille}{rgb}{0.035,0.415,0.035}
\definecolor{vertceladon}{rgb}{0.513,0.65,0.592}
\definecolor{vertchartreuse}{rgb}{0.76,0.968,0.196}
\definecolor{vertdeau}{rgb}{0.69,0.949,0.713}
\definecolor{vertdechrome}{rgb}{0.094,0.223,0.117}
\definecolor{vertdegris}{rgb}{0.584,0.647,0.584}
\definecolor{vertdehooker}{rgb}{0.105,0.309,0.031}
\definecolor{vertdevessie}{rgb}{0.133,0.47,0.058}
\definecolor{vertemeraude}{rgb}{0.003,0.843,0.345}
\definecolor{vertepinard}{rgb}{0.09,0.341,0.196}
\definecolor{vertgazon}{rgb}{0.227,0.615,0.137}
\definecolor{vertimperial}{rgb}{0.0,0.337,0.105}
\definecolor{vertjade}{rgb}{0.529,0.913,0.564}
\definecolor{vertkaki}{rgb}{0.474,0.537,0.2}
\definecolor{vertlichen}{rgb}{0.521,0.756,0.494}
\definecolor{vertlime}{rgb}{0.619,0.992,0.219}
\definecolor{vertmalachite}{rgb}{0.121,0.627,0.333}
\definecolor{vertmeleze}{rgb}{0.219,0.435,0.282}
\definecolor{vertmenthe}{rgb}{0.086,0.721,0.305}
\definecolor{vertmenthealeau}{rgb}{0.329,0.976,0.552}
\definecolor{vertmilitaire}{rgb}{0.349,0.4,0.262}
\definecolor{vertmousse}{rgb}{0.403,0.623,0.352}
\definecolor{vertolive}{rgb}{0.439,0.552,0.137}
\definecolor{vertopaline}{rgb}{0.592,0.874,0.776}
\definecolor{vertperroquet}{rgb}{0.227,0.949,0.294}
\definecolor{vertpin}{rgb}{0.003,0.474,0.435}
\definecolor{vertpistache}{rgb}{0.745,0.96,0.454}
\definecolor{vertpoireau}{rgb}{0.298,0.65,0.419}
\definecolor{vertpomme}{rgb}{0.203,0.788,0.141}
\definecolor{vertprairie}{rgb}{0.341,0.835,0.231}
\definecolor{vertprintemps}{rgb}{0.0,1.0,0.498}
\definecolor{vertsapin}{rgb}{0.035,0.321,0.156}
\definecolor{vertsauge}{rgb}{0.407,0.615,0.443}
\definecolor{vertsecondaire}{rgb}{0.0,1.0,0.0}
\definecolor{verttilleul}{rgb}{0.647,0.819,0.321}
\definecolor{vertturquoise}{rgb}{0.121,0.996,0.847}
\definecolor{vertveronese}{rgb}{0.352,0.396,0.129}
\definecolor{violet}{rgb}{0.4,0.0,0.6}
\definecolor{violetdeveque}{rgb}{0.447,0.243,0.392}
\definecolor{violetdebayeux}{rgb}{0.407,0.133,0.27}
\definecolor{violine}{rgb}{0.631,0.023,0.517}
\definecolor{viride}{rgb}{0.25,0.509,0.427}
\definecolor{zinzolin}{rgb}{0.423,0.007,0.466}
%%%%%%%% code inclusion:
\usepackage{listings}
\usepackage{setspace}

\definecolor{Color_Background}{rgb}{1,1,1}
\definecolor{Color_Code}{rgb}{1,0,0}
\definecolor{Color_Decorators}{rgb}{0.5,0.5,0.5}
\definecolor{Color_Numbers}{rgb}{0.36,0.14,0.43}
\definecolor{Color_Comments}{rgb}{0.11,0.6,0.}
\definecolor{Color_Strings}{rgb}{0.81,0.68,0.}

\definecolor{Color_Keywords_1}{rgb}{0.4,0.4,0.4}
\definecolor{Color_Keywords_2}{rgb}{0.05,0.,0.58}
\definecolor{Color_Keywords_3}{rgb}{1,0.51,0}
\definecolor{Color_Keywords_4}{rgb}{0.78,0.46,0.9}
\definecolor{Color_Keywords_5}{rgb}{0.49,0,0.81}
\definecolor{Color_Operators}{rgb}{1,0,0}


\definecolor{Color_Backquotes}{rgb}{1,0,0}
\definecolor{Color_Classname}{rgb}{1,0,0}
\definecolor{Color_FunctionName}{rgb}{1,0,0}
\definecolor{Color_Matching_Brackets}{rgb}{0.25,0.5,0.5}

%%%%%%%%%%%%%%%%%%%%%%%%% PYTHON

%%%%%%% changes color for digits

% from https://tex.stackexchange.com/questions/34896/coloring-digits-with-the-listings-package
% second answer:
\newcommand\digitstyle{\color{Color_Numbers}}
\makeatletter
\newcommand{\ProcessDigit}[1]
{%
  \ifnum\lst@mode=\lst@Pmode\relax%
   {\digitstyle #1}%
  \else
    #1%
  \fi
}
\makeatother
\lstset{literate=
    {0}{{{\ProcessDigit{0}}}}1
    {1}{{{\ProcessDigit{1}}}}1
    {2}{{{\ProcessDigit{2}}}}1
    {3}{{{\ProcessDigit{3}}}}1
    {4}{{{\ProcessDigit{4}}}}1
    {5}{{{\ProcessDigit{5}}}}1
    {6}{{{\ProcessDigit{6}}}}1
    {7}{{{\ProcessDigit{7}}}}1
    {8}{{{\ProcessDigit{8}}}}1
    {9}{{{\ProcessDigit{9}}}}1
    {<=}{{\(\leq\)}}1,
    morestring=[b]",
    morestring=[b]',
    morecomment=[l]//,
}


\lstset{
language=Python,
                                            %
%numbering
numbers=left,       % where to put the line-numbers; possible values are (none, left, right)
numberstyle=\footnotesize,     % the style that is used for the line-numbers
numbersep=1em,     % how far the line-numbers are from the code
stepnumber = 1, % the step between two line-numbers. If it's 1, each line will be numbered
                                            %
%font style
xleftmargin=1em,
framextopmargin=2em,
framexbottommargin=2em,
showspaces=false,                % show spaces everywhere adding particular underscores; it overrides 'showstringspaces'
showtabs=false,             % show tabs within strings adding particular underscores
showstringspaces=false,       % underline spaces within strings only
frame=l,       % adds a frame around the code
tabsize=4,      % sets default tabsize to x spaces
breaklines=true,                            % allows for breaking lines of the code
breakatwhitespace = false,                  % sets if automatic breaks should only happen at whitespace
                                            %
% Basic
basicstyle=\ttfamily\tiny\setstretch{1},    % the size of the fonts that are used for the code
backgroundcolor=\color{Color_Background},         % choose the background color; you must add \usepackage{color} or \usepackage{xcolor}
                                            %
% Comments
commentstyle=\color{Color_Comments},      % comment style
                                            %
% Strings
%stringstyle=\color{Color_Strings},                  %string style
                                            %
% DocStrings
% I was not able to successfully untangled comments from docstrings.
%moredelim=**[s][\color{gray}]{"""}{"""},    %adding the """
%moredelim=**[s][\color{gray}]{'''}{'''},    %adding the '''
                                            %
% Operators
%otherkeywords={!,!=,~,$,*,\&,+,-,^,\%,\%/\%,\%*\%,\%\%,<-,<<-,/},
%keywordstyle=\color{Color_Keywords_2},
                                            %
                                            %
                                            %
% keywords_1 : BOOL
                                            %
emph={fsdjskq}, % add keywords to a list
emphstyle={\color{Color_Keywords_1}},      % keyword style
                                            %
                                            %
% keywords_1 : BOOL
                                            %
emph={[10]False,True,None}, % add keywords to a list
emphstyle={[10]\color{Color_Keywords_1}},      % keyword style
                                            %
                                            %
% keywords_2 : CLASSICS
                                            %
emph={[2]import,include,from,def,for,
while,if,is,in,elif,else,not,and,or,print,break,
continue,return,as,del,except,exec,global,from,
finally,global,import,lambda,inline,pass,print,
Exception,raise,try,assert}, % add keywords to a list
emphstyle={[2]\color{Color_Keywords_2}\bfseries},      % keyword style
                                            %
                                            %
% keywords_3 : TYPES AND UTILITIES
                                            %
emph={[3]self,class,object,type,isinstance,copy,deepcopy,
zip,enumerate,list,set,len,dict,tuple,range,
append,execfile,real,imag,reduce,str,repr}, % add keywords to a list
emphstyle={[3]\color{Color_Keywords_3}\bfseries},      % keyword style
                                            %
                                            %
% keywords_4 : ALG LIB
                                            %
emph={[4]ode,fsolve,sqrt,exp,sin,cos,
pi,array,norm,solve,dot,arange,linspace,isscalar,max,
sum,flatten,shape,reshape,find,any,all,abs,plot,legend,quad,
polyval,polyfit,hstack,concatenate,vstack,column_stack,
empty,zeros,ones,rand,vander,grid,pcolor,eig,eigs,
eigvals,svd,qr,tan,det,logspace,roll,min,mean,cumsum,cumprod,
diff,vectorize,lstsq,cla,eye,xlabel,ylabel,squeeze}, % add keywords to a list
emphstyle={[4]\color{Color_Keywords_4}},      % keyword style
                                            %
                                            %
% keywords_5 : CLASS FCT
                                            %
emph={[5]__init__,__add__,__mul__,__div__,__sub__,__call__
,__getitem__,__setitem__,__eq__,__ne__,__nonzero__,__rmul__,
__radd__,__repr__,__str__,__get__,__truediv__,__pow__,__name__
,__future__,__all__}, % add keywords to a list
emphstyle={[5]\color{Color_Keywords_5}\bfseries},      % keyword style
                                            %
                                            %
% simple import decorators etc not so important for code!
emph={@invariant,pylab,numpy,np,scipy,plt,math,bisect,tqdm},
emphstyle={\color{Color_Decorators}\slshape},
}



















%%%%%%%%%%%%%%%%%%%%%%%%%%% TODO
\usepackage{xargs}                      % Use more than one optional parameter in a new commands
\usepackage[colorinlistoftodos,prependcaption,textsize=footnotesize]{todonotes}

%red
\newcommandx{\willdo}[1]{\todo[linecolor=orangebrulee,backgroundcolor=orangebrulee!35,bordercolor=orangebrulee,inline]{WILL DO : #1}}
%green
\newcommandx{\willprecise}[1]{\todo[linecolor=blue,backgroundcolor=blue!25,bordercolor=blue,inline]{WILL PRECISE : #1}}
%purple
\newcommandx{\willquestion}[1]{\todo[linecolor=OliveGreen,backgroundcolor=OliveGreen!25,bordercolor=OliveGreen,inline]{QUESTION : #1}}

\newcommandx{\willlastcheck}[1]{\todo[linecolor=SpringGreen,backgroundcolor=SpringGreen!25,bordercolor=SpringGreen,inline]{WILL LAST CHECK BEFORE SUBMIT :#1}}

\newcommandx{\thiswillnotshow}[1]{\todo[disable,inline]{#1}}
%%%%%%%%%%%%%%%%%%%%%%%%%%%%%%%%%%%%%%

















%%%%%%%%%%%%%%%%%%%%%%%%%
%%%%% Maths Symbols %%%%%
%%%%%%%%%%%%%%%%%%%%%%%%%


%declare operator%https://tex.stackexchange.com/questions/67506/newcommand-vs-declaremathoperator
% diff * and no star: https://tex.stackexchange.com/questions/1050/whats-the-difference-between-newcommand-and-newcommand
% for allowing subscript under the operator use \DeclareMathOperator* instead of \DeclareMathOperator

%\newcommand{•}{•}
%\renewcommand{•}{•}




%---- Ensemles : entiers, reels, complexes... ----
\newcommand*{\N}{\mathbb{N}_{\geq 0}}
\newcommand*{\Z}{\mathbb{Z}}
\newcommand*{\Q}{\mathbb{Q}}
\newcommand*{\R}{\mathbb{R}}
\newcommand*{\C}{\mathbb{C}}
\newcommand*{\Part}{\mathcal{P}}

\newcommand*{\abs}[1]{\lvert #1\rvert}
\newcommand*{\norm}[1]{\lVert #1\rVert}
\newcommand*{\anglebrackets}[1]{\left\langle #1 \right\rangle}
\newcommand*{\curlybrackets}[1]{\left \{ #1 \right \}}
\DeclareMathOperator{\MSE}{MSE}
\DeclareMathOperator{\MISE}{MISE}
\DeclareMathOperator{\sign}{Sign}

%\newcommand{\systeme}[1][2]{ \left\{ \begin{array}{#1}#2\end{array} \right. }}

\newcommand{\Tau}{\mathcal{T}} % big tau


\newcommand*{\Mat}{\mathrm{Mat}}
\DeclareMathOperator{\fix}{fix}
\DeclareMathOperator{\End}{End}        %Endomorphismes
\DeclareMathOperator{\Hom}{Hom}
\DeclareMathOperator{\Id}{Id}
\DeclareMathOperator{\image}{image}
\DeclareMathOperator{\im}{Im}
\DeclareMathOperator{\tr}{tr}
\DeclareMathOperator{\Tr}{Tr}
\newcommand*{\Bilin}{\mathrm{Bilin}}
\newcommand*{\Vect}{\mathrm{Vect}}
\newcommand*{\Ker}{\mathop{\mathrm{Ker}}\nolimits}
\newcommand*{\rg}{\mathop{\mathrm{rg}}\nolimits}
\newcommand*{\Sp}{\mathsf{Sp}}
\newcommand*{\dx}{\partial_x}
\newcommand*{\dy}{\partial_y}


%fonction charactéristique
\DeclareMathOperator{\11charac}{\mathds{1}}

%arg max and min
\DeclareMathOperator*{\argmax}{\arg\!\max}
\DeclareMathOperator*{\argmin}{\arg\!\min}

%probability :
\newcommand{\E}{ \mathbb E }
\newcommand{\Var}{\mathrm{Var}}
\newcommand{\Cov}{\mathrm{Cov}}




%---- Opérateurs utiles ----
% number theory dividing side 6 is divisible 3
\newcommand*\divise{\mathrel{\mid}}
\renewcommand{\bigoplus}{\mathop{\hbox{\large $\oplus$}}}
\newcommand*{\Bigoplus}{\mathop{\hbox{\Large $\oplus$}}}

\DeclareMathOperator{\Non}{non}

\newcommand{\pgcd}{\text{pgcd}}
\newcommand{\ppcm}{\text{ppcm}}
\DeclareMathOperator*{\Max}{Max}
\DeclareMathOperator*{\Min}{Min}
\DeclareMathOperator*{\Sup}{Sup}
\DeclareMathOperator*{\Inf}{Inf}
\DeclareMathOperator*{\Card}{Card}

\DeclareMathOperator*{\supp}{supp}


\newcommand*\conj[1]{%
   \hbox{%
     \vbox{%
       \hrule height 0.5pt % The actual bar
       \kern0.5ex%         % Distance between bar and symbol
       \hbox{%
         \kern-0.1em%      % Shortening on the left side
         \ensuremath{#1}%
         \kern-0.1em%      % Shortening on the right side
       }%
     }%
   }%
}


\newcommand*\mean[1]{%
   \hbox{%
     \vbox{%
       \hrule height 0.5pt % The actual bar
       \kern0.5ex%         % Distance between bar and symbol
       \hbox{%
         \kern-0.1em%      % Shortening on the left side
         \ensuremath{#1}%
         \kern-0.1em%      % Shortening on the right side
       }%
     }%
   }%
}



%% eviter les boucles infinies :
\let\oldforall\forall
\renewcommand{\forall}{\oldforall \, }

\let\oldexist\exists
\renewcommand{\exists}{\oldexist \: }

\newcommand\existu{\oldexist! \: }












%---- Démonstrations ----
\newcommand*{\sensdirect}{\par\smallbreak\mbox{$(\implies)$}\xspace}
\newcommand*{\sensindirect}{\par\smallbreak\mbox{$(\impliedby)$}\xspace}
\newcommand*{\implication}[2]{\par\smallbreak\mbox{(\romannumeral#1)$\implies$(\romannumeral#2)}}

\newcommand\analyse{\mbox{\sl \textbf{Analyse: }}\xspace}
\newcommand\synthese{\mbox{\sl \textbf{Synthèse: }}\xspace}
\newcommand\conclusion{\mbox{\sl \textbf{Conclusion: }}\xspace}

\newcommand\existence{\mbox{\sl \textbf{Existence: }}\xspace}
\newcommand\unicite{\mbox{\sl \textbf{Unicité:}}\xspace}


\newcommand\casparticulier{\mbox{\sl \textbf{Cas particulier: }}\xspace}
\newcommand\heredite{\mbox{\sl \textbf{Hérédité: }}\xspace}




\newcommand\cas[1]{\mbox{{\sl \textbf{Cas:}\xspace #1.}}\xspace}
\newcommand\casgeneral{\mbox{\sl \textbf{Cas général:}}\xspace}







% do not trigger any warning related to hbox. I.E. that the line is too long. I had plenty of such useless warnings. 
\hbadness = 15000
\hfuzz = 100 pt 
\vbadness=\maxdimen